\documentclass[bimj,fleqn]{w-art}
\usepackage{times}
\usepackage{w-thm}
\usepackage[authoryear]{natbib}
\usepackage{amsfonts}
% \setlength{\bibsep}{2pt}
% \setlength{\bibhang}{2em}
% \newcommand{\J}{J\"{o}reskog}
% \newcommand{\So}{S\"{o}rbom}
% \newcommand{\bcx}{{\bf X}}
% \newcommand{\bcy}{{\bf Y}}
% \newcommand{\bcz}{{\bf Z}}
% \newcommand{\bcu}{{\bf U}}
% \newcommand{\bcv}{{\bf V}}
% \newcommand{\bcw}{{\bf W}}
% \newcommand{\bci}{{\bf I}}
% \newcommand{\bch}{{\bf H}}
% \newcommand{\bcb}{{\bf B}}
% \newcommand{\bcr}{{\bf R}}
% \newcommand{\bcm}{{\bf M}}
% \newcommand{\bcf}{{\bf F}}
% \newcommand{\bcg}{{\bf G}}
% \newcommand{\bcs}{{\bf S}}
% \newcommand{\bca}{{\bf A}}
% \newcommand{\bcd}{{\bf D}}
% \newcommand{\bcc}{{\bf C}}
% \newcommand{\bce}{{\bf E}}
% \newcommand{\ba}{{\bf a}}
% \newcommand{\bb}{{\bf b}}
% \newcommand{\bc}{{\bf c}}
% \newcommand{\bd}{{\bf d}}
% \newcommand{\bx}{{\bf x}}
% \newcommand{\by}{{\bf y}}
% \newcommand{\bz}{{\bf z}}
% \newcommand{\bu}{{\bf u}}
% \newcommand{\bv}{{\bf v}}
% \newcommand{\bh}{{\bf h}}
% \newcommand{\bl}{{\bf l}}
% \newcommand{\be}{{\bf e}}
% \newcommand{\br}{{\bf r}}
% \newcommand{\bw}{{\bf w}}
% \newcommand{\de}{\stackrel{D}{=}}
% \newcommand{\bt}{\bigtriangleup}
% \newcommand{\bfequiv}{\mbox{\boldmath $\equiv$}}
% \newcommand{\bmu}{\mbox{\boldmath $\mu$}}
% \newcommand{\bnu}{\mbox{\boldmath $\nu$}}
% \newcommand{\bxi}{\mbox{\boldmath $\xi$}}
% \newcommand{\btau}{\mbox{\boldmath $\tau$}}
% \newcommand{\bgamma}{\mbox{\boldmath $\Gamma$}}
% \newcommand{\bphi}{\mbox{\boldmath $\Phi$}}
% \newcommand{\bfphi}{\mbox{\boldmath $\varphi$}}
% \newcommand{\bfeta}{\mbox{\boldmath $\eta$}}
% \newcommand{\bpi}{\mbox{\boldmath $\Pi$}}
% \newcommand{\bequiv}{\mbox{\boldmath $\equiv$}}
% \newcommand{\bvarepsilon}{\mbox{\boldmath $\varepsilon$}}
% \newcommand{\btriangle}{\mbox{\boldmath $\triangle$}}
% \newcommand{\bdelta}{\mbox{\boldmath $\Delta$}}
% \newcommand{\beps}{\mbox{\boldmath $\epsilon$}}
% \newcommand{\btheta}{\mbox{\boldmath $\theta$}}
% \newcommand{\balpha}{\mbox{\boldmath $\alpha$}}
% \newcommand{\bsphi}{\mbox{\boldmath $\varphi$}}
% \newcommand{\bsig}{\mbox{\boldmath $\sigma$}}
% \newcommand{\bfpsi}{\mbox{\boldmath $\psi$}}
% \newcommand{\bfdelta}{\mbox{\boldmath $\delta$}}
% \newcommand{\bsigma}{{\bf \Sigma}}
% \newcommand{\bzero}{{\bf 0}}
% \newcommand{\bpsi}{\mbox{\boldmath $\Psi$}}
% \newcommand{\bep}{\mbox{\boldmath $\epsilon$}}
% \newcommand{\bomega}{\mbox{\boldmath $\Omega$}}
% \newcommand{\bfomega}{\mbox{\boldmath $\omega$}}
% \newcommand{\blambda}{\mbox{\boldmath $\Lambda$}}
% \newcommand{\bflambda}{\mbox{\boldmath $\lambda$}}
% \newcommand{\bfsigma}{\mbox{\boldmath $\sigma$}}
% \newcommand{\bfpi}{{\mbox{\boldmath $\pi$}}}
% \newcommand{\bupsilon}{\mbox{\boldmath $\upsilon$}}
% \newcommand{\obs}{{\rm obs}}
% \newcommand{\mis}{{\rm mis}}
% \theoremstyle{plain}
% \newtheorem{criterion}{Criterion}
% \theoremstyle{definition}
% \newtheorem{condition}[theorem]{Condition}
% \usepackage[]{graphicx}
% \chardef\bslash=`\\ % p. 424, TeXbook
% \newcommand{\ntt}{\normalfont\ttfamily}
% \newcommand{\cn}[1]{{\protect\ntt\bslash#1}}
% \newcommand{\pkg}[1]{{\protect\ntt#1}}
% \let\fn\pkg
% \let\env\pkg
% \let\opt\pkg
% \hfuzz1pc % Don't bother to report overfull boxes if overage is < 1pc
% \newcommand{\envert}[1]{\left\lvert#1\right\rvert}
% \let\abs=\envert

\begin{document}
%\DOIsuffix{bimj.DOIsuffix}
\DOIsuffix{bimj.200100000}
\Volume{52}
\Issue{61}
\Year{2010}
\pagespan{1}{}
\keywords{Evaluation; Imputation; Missing data; Simulation studies;\\ [1pc]
% \noindent \hspace*{-4.2pc} Supporting Information for this article is available from Github, \break \hspace*{-4pc} \underline{github.com/gerkovink/StandardizedEvaluation}.
% } %%% semicolon and fullpoint added here for keyword style
\noindent \hspace*{-4pc} Supporting Information for this article is available from \underline{www.gerkovink.com/evaluation}.\\
% \hspace*{-4pc} \underline{gerkovink.com/evaluation}. 
}  %%% semicolon and fullpoint added here for keyword style

\title[Towards a standardized evaluation of imputation methodology]{Towards a standardized evaluation of imputation methodology}
%% Information for the first author.
\author[Oberman]{Hanne I. Oberman\footnote{Corresponding author: {\sf{e-mail: h.i.oberman@uu.nl}}}\inst{,1}} 
\address[\inst{1}]{Departement of Methodology \& Statistics, Padualaan 14, 3584 CH Utrecht, The Netherlands}
%%%%  Information for the second author
\author[Vink]{Gerko Vink\inst{1}}
%%%%  \dedicatory{This is a dedicatory.}
\Receiveddate{zzz} \Reviseddate{zzz} \Accepteddate{zzz} 

\begin{abstract}
Developing new imputation methodology has become a very active field. Unfortunately, there is no consensus on how to perform simulation studies to evaluate the properties of imputation methods. In this paper, we propose a move towards a standardized evaluation of imputation methodology. To demonstrate the need for standardization, we highlight a set of potential pitfalls that bring forth a chain of potential problems in the objective assessment of the performance of imputation routines. This may lead to sub-optimal use of  imputation in practice. Additionally, we suggest a course of action for simulating and evaluating missing data problems. [TODO: Narrowing the scope of the manuscript to the evaluation of methodology for statistical inference, not prediction or causal inference (although inferential validity encompasses predictive validity and may thus be generalized?). Emphasizing that the ideal evaluation of imputation methodology depends on the aim of the specific study (i.e. how the evaluated methods are used in practice). Split up the problems with the simulation setup/evaluation and the problems with the interpretation/extrapolation from these simulation studies. See e.g. \citet{gree17} on cognitive problem of generelization versus methodological problems. Explain how the standardisation leads to more neutrality, but that this is just a set of elements to consider/report, not a complete 'cookbook'. Add that we focus on  simulation studies evaluating imputation methodology for statistical inference.]
\end{abstract}



%% maketitle must follow the abstract.
\maketitle          % Produces the title.

%% If there is not enough space inside the running head
%% for all authors including the title you may provide
%% the leftmark in one of the following three forms:

%% \renewcommand{\leftmark}
%% {First Author: A Short Title}

%% \renewcommand{\leftmark}
%% {First Author and Second Author: A Short Title}

%% \renewcommand{\leftmark}
%% {First Author et al.: A Short Title}

%% \tableofcontents % Produces the table of contents.

%%%%%%%%%%%%%%%%%%%%%%%%%%
%% INTRODUCTION
%%%%%%%%%%%%%%%%%%%%%%%%%%

\section{Introduction}

Imputation is a state-of-the-art technique for drawing valid conclusions from incomplete data. The technique has earned a permanent spot in research and policymaking, demonstrated e.g. by the detailed manual created by the National Research Council \citep*{little2012prevention}. Although top-down enforcement of valid ways to handle missing data is not yet very pronounced, an increasing amount of researchers and data scientists are embracing imputation techniques. After all, the principle of imputation is very intuitive. The idea behind imputation is to impute (fill in) missing values, to obtain a valid estimate of what could have been. The completed data that are thus obtained can be analyzed by standard techniques, which conveniently separates the missing data problem from the analysis. When prediction is the goal, a single imputation may be sufficient \citet{sper20}. For inferential analyses, multiple imputation \citep{rubi76} has proven to be a valuable technique for obtaining valid inferences on incomplete datasets. In the case of multiple imputation each missing datum is imputed more than once, leading to multiple completed data sets. The resulting multiple inferences can be combined into a single inference using Rubin's rules \citep[][p. 76]{rubi87}. The quality of a solution obtained by imputation depends on the statistical properties of the incomplete data and the degree to which an imputation procedure is able to capture these properties when modeling missing values. In general, it holds that modeling missing data becomes more challenging when the amount of missingness increases. However, when (strong) relations in the data are present, the observed parts can hold great predictive power for the models that estimate the missingness. In that case, imputation would be substantially more efficient than the ubiquitous complete case analysis.

% In contrast to ad hoc methods for dealing with missing values (e.g. list-wise deletion), imputation allows analysts to account for the sources of uncertainty that are related to the missingness problem. With the popular variant multiple imputation \citep{rubi76}, each missing value is replaced by a plausible data value several times, resulting in multiple completed data sets. The estimated parameters of interest for each of these completed sets can be combined into a single inference using Rubin's rules \citep[][p. 76]{rubi87}. [TODO: add that if the goal is statistical inference, the uncertainty due to missingness should be quantified/captured/propagated by the missing data method. Refer to \citet{sper20} for the context of prediction, where multiple imputation is not always necessary.] [TODO: paraphrase ``the quality of a technique can only be evaluated with respect to the aims of the problem/s it's intended to solve in practice".] 

When evaluating the statistical properties (and thereby the practical applicability) of imputation methodology, researchers most often make use of simulation studies. Such studies are typically composed of an analysis pipeline in which data are generated, made incomplete, and imputed repeatedly under different simulation conditions. A set of evaluation criteria is then postulated to evaluate the performance of one or more missing data methods. However, there is no consensus on how to design, execute and report simulation studies aimed to evaluate imputation methodology. Without a `gold standard' for the evaluation of imputation routines, the validity and comparability of simulation setups may differ tremendously from one developer to another. This brings forth a chain of potential problems in the objective assessment of imputation method performance within and across studies, which may lead to sub-optimal use of imputation in practice.

% Especially with novel imputation methods being propagated from the fields of machine learning and artificial intelligence, the differences may become more pronounced. Although these promising new methods seem to yield even sharper imputations than now-standard imputation methods, the comparison may not be fair due to the simulation setup. 

The purpose of this paper is threefold: First, to raise some concerns with respect to evaluating imputation methodology. These concerns stem from careful consideration with fellow `imputers' and from encounters as a reviewer for statistical journals. Second, to provide imputation methodologists with a suggested course of action when using simulation studies to evaluate imputation techniques for missing data problems. This suggested approach should identify common ground but is in no way intended as an absolute solution. This identifies the third purpose of this paper: discussion. We hope to elicit critical thinking regarding the problems at hand. We are all convinced that our methodology has some merit. But for sake of progress, it would be much more advantageous if the aim of our evaluations would go beyond \textit{proving the point} and would legitimately consider the statistical properties. 

Standardizing the evaluation of imputation methodology requires simulators to consider common aspects in their simulation workflows. Table \ref{table:check} outlines some suggested steps to adopt. Please note that there is--of course--no `silver bullet' and that we do not claim to present a universally applicable approach. Our suggestions are the accumulated result of scientific literature, research experience, and discussion. For an excellent overview of general best practices for method evaluation by means of simulation, see \citet{morr18}. We recommend adhering to their proposed ADEMP structure (aims, data-generating mechanisms, estimands, methods, performance measures) for planning and reporting simulation studies, but highlight and add some specific recommendations for the evaluation of imputation methodology. [TODO: Add advice to pre-register simulation protocols, which allows for fair comparisons of per-protocol results \citep{pawe22}.]

%%%%%%%%%%%%%%%%%%%%%%%%%%
%% PROBLEMS
%%%%%%%%%%%%%%%%%%%%%%%%%%

\section{Why some evaluations should not be trusted}

% [TODO: split up into design, execution and reporting? \citep{pawe22}]

The ideal evaluation of imputation methodology depends on the aim of the specific study (i.e. how the evaluated methods are used in practice). A simulation study developed for the comparison of imputation methods will typically have a different design than one aimed at establishing the inferential validity of a single (novel) method. The suitability of the evaluations should be assessed with the imputation methods' purpose in mind, since the simulation aims and setup are naturally intertwined with the choice of imputation method(s) under evaluation.

We limit the scope of this paper to comparative simulation studies in the context of statistical inference. We thereby exclude the evaluation of imputation methodology for prediction and causal inference. Simulation study designs aimed at comparing different methods' predictive performance in incomplete data typically do not start out from a complete data set in which missingness is induced by the simulator \citep{liu21}. Rather, the methods are evaluated against the observed values of a certain outcome variable (or target) as the estimand. After imputing the missingness in the incomplete predictor (or feature) space, a prediction method is applied to estimate the outcome. Pairs of imputation and prediction methods with high predictive accuracy in one or more benchmark data sets are deemed as `good'. Note that such a simulation design does not have a `ground truth', so only the comparative performance of methods may be established. We do not recommend this approach if the inferential validity of the imputations is of interest. For an overview of missingness in the prediction context see e.g. \citet{sper20}. Missing data within a causal inference framework has been described by e.g. \cite{more18} and \cite{moha21}.

Within the scope of comparative simulation studies aimed at statistical inference, we observe several problems. To demonstrate the broad impact of these problems, we recognize the following four distinct categories: problems with simulation design, problems with data generation, problems with missingness generation, and problems with performance evaluation. We further detail the impact each of these problems may have on the validity of evaluations. 


%%%%%%%%%%%%%%%%%%%%%%%%%%

\subsection{Simulation design}


% Before setting up a simulation study, the simulator should clearly define the scope of their evaluations. 
Estimand(s) or other simulation targets should be defined in the context of the study aim. [TODO: refer to \citet{pete14} about the need for a clear and unambiguous estimand.] 

[TODO: describe problems with unclear simulation parameters, extrapolation beyond scope of simulations, QRPs \citep{pawe22} and misinterpretation \citep[see "investigator bias"][]{gree17}.]

[TODO: rewrite sampling variance paragraphs and combine with ``If there is more than one imputation method under evaluation, the simulation design should apply each method to every incomplete data set. Applying all methods to the same incomplete data set is computationally convenient, and minimizes unnecessary variation, which makes for fairer comparisons."]

[TODO: add nuance because this doesn't always hold, see Tims review!] Simulation studies on missing data methodology have the unique option to exclude sampling variance from their evaluations and only use the missing data generation procedure as the source of Monte Carlo variation. After all, we are interested in the missingness and are not considering the noise induced by the sampling mechanism for evaluation in such studies. Therefore, taking sampling variation into account is neither necessary for obtaining information about a method's ability to handle the missing data problem, nor for objectively comparing methods on their ability to correct for missingness \citep[see for a detailed discussion][]{vink14}. The only required change to simulation setups is to remove sampling variance from performance evaluation. Still, a lot of published work does not consider this option, while it could have sharpened inconclusive or obfuscated results. The comparative truth of the estimand(s) is determined by the choice of data-generating mechanism and whether sampling variance is included in the simulation setup, see Table \ref{table:dgm_truth}. If sampling variance cannot be omitted from the simulation scheme, multiple samples should be drawn from the data-generating mechanism (i.e., one sample per simulation repetition, a standard simulation setup). If sampling variance is not of interest, a single complete data set can be obtained from the data-generating mechanism (i.e., one sample for all simulation repetitions). This process is computationally convenient because only a single complete data set has to be considered during all of the simulations. We highly value the additional flexibility that such a simulation setup offers. Especially in the case of data transformations, it can be challenging to derive the true parametric references to evaluate against. Simply using a single generated complete set in which missingness is induced, and evaluating against that generated data set avoids a plethora of procedural problems and computational challenges.
When deviating from the standard simulation workflow, conventional pooling rules for multiple imputation \citep[cf.][p. 76-77]{rubi87} do not apply. Instead, alternative pooling rules need to be used \citep{raghunathan2003multiple,vink14}.

\begin{table}[tb]
\begin{center}
\caption{Source of the comparative truth in simulation studies.}
\label{table:dgm_truth}
\begin{tabular}{lll}
\hline
               & With sampling variance      & Without sampling variance \\
               & (multiple samples drawn)    & (one sample drawn) \\
\hline  
Model-based simulation   & data-generating model         & single sample \\
Design-based simulation  & sufficiently large data set   & single sample \\
\hline
\end{tabular}
\end{center}
\end{table}

The simulation design should ideally be fully factorial, i.e. varying each simulation condition against all other conditions. There may, for example, be interactions between the source and amount of missingness in the incomplete data and the efficacy of imputation methods (i.e., the validity of the assumed missingness mechanism becomes increasingly important with higher missingness proportions). The number of simulation repetitions may be informed by the required level of precision in the simulation study \citep[e.g. as determined from a maximum tolerable level of uncertainty in terms of a performance measure's Monte Carlo error][]{morr18}. 


%%%%%%%%%%%%%%%%%%%%%%%%%%

\subsection{Data generation}

% [TODO: split up into data and missingness part of the DGM, with choices between data sources, missingness mechanisms and proportions. And divide problems between simulation and interpretation, see Reviewer 2: "whether method A (e.g. imputation) outperforms method B (e.g. listwise deletion) depends on the estimand, purpose of the study, used distributions (in the DGP, imputation model, possibly also analysis model), and simulation setup (e.g. dependency structure) - if a human misinterprets findings in a comparison of method A and B, then this is a cognitive problem and not the problem on how missingness was generated in the DGP."] 

[TODO; describe problems with data generation and interpretation: If the source of the data is model-based, the correlations in the data may be so high that there is no effective missingness due to the predictive power in the other variables, and the data source may be unfair if the data generating model is equal to one of the imputation models under evaluation. With a design-based simulation, the data source is unlikely to have no missingness and the choice of initial missing data method may propagate into the evaluations.]

To evaluate the ability of an imputation routine to handle missingness, a form of truth has to be established. Those who perform simulation studies are in the luxury position to establish the truth beforehand by choosing a data-generating mechanism. Data-generating mechanisms define how a complete dataset is obtained at the start of each simulation repetition. There are two general approaches to generating complete data: (i) model-based simulation, in which data are drawn from a known statistical model or probability distribution, such as the multivariate normal distribution; and (ii) design-based simulation, where data are sampled (with replacement) from a sufficiently large observed set, such as official registers. The simulator should choose their data-generating mechanism(s) in line with the study scope. Model-based data-generating mechanisms have advantages in flexibility and precision, since data are generated from a known statistical model and the true theoretical parameters can be derived. A design-based approach is often used in situations where a probability distribution is not available, or where real-life data structures are of interest.

The problem with model-based data-generating mechanisms is that a method's performance on simulated data may not translate to empirical data. Real-life data hardly ever follow a given theoretical distribution, so there is no guarantee that simulation results are generalizable. Moreover, data are often generated such that the problem being studied is most pronounced, e.g. with consistently high correlations between groups of variables. This results in simulated data that contain such valuable information structures that, no matter what type of missingness would subsequently be induced, the observed parts of the data will still hold much (if not all) of the information about the missing part. Unsurprisingly, the performance of any imputation method will then be evaluated as good. Another threat to the generalizability of model-based simulations is the use of a single model for both data generation and imputation. If data are generated following a model that is also used for imputing the data, the imputation approach will be deemed good (or better than other methods) purely due to the evaluated conditions being in favor of the problem that is studied. Other (unfair) comparative advantages in favor of a certain imputation method may occur due to characteristics of the generated data, such as the number of observations, the number of variables, the variable type(s), and the coherence between variables. In contrast to design-based studies, such characteristics are not always explicit simulation conditions, which may give a false sense of objectivity.

An obvious problem with design-based simulation is that obtaining a large dataset without missing entries can be very challenging. Most real-world data contains at least some missing entries, for which the true underlying missing data model is--by definition--unknown. Therefore, the simulator needs to deal with missingness in \textit{some} way before incomplete empirical data can serve as comparative truth in the simulations. It may seem like an intuitive solution to only draw complete cases from the large dataset, which would indeed yield complete samples. However, there may be inherent differences between cases with and cases without any missing values, due to the unknown missing data model. Only sampling complete cases from the data set may thus result in samples that fail to capture all relevant real-world conditions, which in turn refutes the main reason for using design-based simulation. Another way to deal with missingness in a design-based simulation is to impute the incomplete dataset once, to obtain a single completed dataset to draw samples from. Unfortunately, this practice may favor the imputation method that was used in this initial imputation step throughout any further evaluations. Just to be clear: leaving the missingness as-is and inducing additional missing values to impute is no option here, because we would not have a real and unbiased comparative truth.


%%%%%%%%%%%%%%%%%%%%%%%%%%

% 1a. choose DGM (if model-based: choose n_obs, n_var, relations between var, var types, etc; if design-based: choose n_obs, missing data handling strategy)
% 1b. decide if sampling variance is needed
% 1c. draw complete sample(s)


%%%%%%%%%%%%%%%%%%%%%%%%%%

\subsection{Missingness generation}

[TODO: Make clear in this section that missingness mechanisms can be spurious (spurious MAR when correlations are too low, or when missingness proportion is too low). In reporting, the functional form or type of missingness is often omitted. As advice, MCAR can be a very informative mechnism for reference and MNAR may be useful for practice. Missingness proportions can be misinterpreted. Missingness pattern may have unexpected consequences (e.g., univariately generated multivariate missingness). In interpretation, watch out for generalizations from unintended special cases.]

% Evaluating imputation methodology requires a missing data problem to be solved. After establishing a comparative truth from a data-generating mechanism, some form of missingness therefore has to be induced.

Often reports of simulation studies remain vague about the missingness conditions under investigation and, even worse, some authors only report something like:
\begin{quote}
\textit{We generated missing data following a missing at random missingness mechanism.}
\end{quote}
This should be considered unacceptable, as claims about the validity of the imputation inference heavily depend on the simulated missingness conditions, such as missingness mechanisms and missingness patterns. Missingness mechanisms describe the relationship between missing entries and observed data values whereas missingness patterns concern the location of missing entries across incomplete data \citep[][p. 8]{litt20}. Under this definition, there is a row-wise element to the missingness pattern describing which variables are jointly observed, and a column-wise element encompassing the amount of missingness in the data. 

% \begin{table}[tb]
% \begin{center}
% \caption{Missingness mechanisms. [TODO: add refs \citet{seam13, meal15, dore18} for clarification and alternatives outlined in \citet{moha21, scho21, more18}]}
% \label{table:mech}
% \begin{tabular}{lll}
% \hline
%       & Interpretation and consequences of the mechanism \\
% \hline  
% MCAR  & Missing completely at random: The probability to be missing is the same for all \\
%       & cases. In other words, the missing values are missing at random and the observed \\
%       & values are observed at random. \\ \noalign{\smallskip}
% MAR   & Missing at random: The probability to be missing is the same within groups of \\
%       & cases defined by the observed data only. The essence is that observed relations in \\
%       & the data inform the missingness, such that MCAR may be assumed within the \\ 
%       & observed groups. \\ \noalign{\smallskip}
% v-MAR & Conditional on the fully observed variables missingness occurs at random 
%       & \citep[][p. 1026]{moha21}.\\
% MNAR  & Missing not at random: The probability to be missing depends on unobserved \\
%       & aspects of the data. The cause of the missingness is unknown and cannot be \\ 
%       & inferred from the observed data. This is considered non-ignorable missingness \\
%       & \citep[see e.g.][]{rubi76}. \\
% \hline
% \end{tabular}
% \end{center}
% \end{table}

Even the terminology on missingness generation can be confusing. Does a missingness proportion of 50\% mean that half of the entries in an incomplete dataset are missing, or that half of the rows have at least one missing entry? In this paper, we will henceforth refer to the latter as the proportion of incomplete cases, and keep the term `missingness proportion' restricted to the variable-by-variable interpretation. The distinction between the two concepts, however, is not always clear in the literature, and convoluting the terms may lead to incorrect generalizations because they rarely mean the same (e.g., a proportion of 50\% incomplete cases in bivariate data could translate to a missingness proportion of 25\% in both variables, or one completely observed variable and one with 50\% missingness). Simulators should therefore be explicit in their description of the missing data generating model fitting the aim of their simulations.

% Moreover, the value of the actual missingness proportion may be diffusing too. Some studies use only 10\% missingness whereas other studies push the limits to additionally investigate the performance under missingness proportions of at least 50\%. The inconsistent display of simulation conditions may impact the objectivity of meta-evaluations over imputation methods, as one method's performance may appear to be favorable because of less stringent simulation conditions. This ultimately may lead to statisticians recommending a less efficient method to applied researchers, thereby limiting the efficiency of the imputation approach and unnecessarily lowering the statistical power.

% Another aspect of missingness patterns that may misguide evaluations is the complexity of the patterns across variables. It is hardly reasonable to imagine empirical data with only one incomplete variable, yet some simulation studies rely on univariate missingness patterns anyways.

Extrapolating simulation results from a specific missing data generating model to more intricate (empirical) missingness could lead to sub-optimal advice in practice. For example, if a simulator induces missingness in the outcome variable of their analysis model exclusively, they may inadvertedly induce missingness according to the special case descibed by \citet[][p. \S 2.7]{buur18}. Under this specific missingness model, list-wise deletion may outperform any imputation method. Such a conclusion would, unfortunately, only translate to data that adhere to the same special case, while biasing inferences in other cases. % Another example of a missingness pattern that may inadvertently impose assumptions on generalizations is a monotone pattern \citep[i.e. a pattern with uniformly increasinmissingness proportions along variables;][]{litt20}. If a simulator investigates iterative imputation methods under monotone missingness, they might believe that the evaluated imputation algorithms converge instantly, while these methods actually require iteration to produce valid results in all other situations. 
Another example of potential side-effects of missingness generation may occur when multivariate missingness is generated using step-wise univariate missingness induction. The resulting incomplete data may then not have the desired statistical properties \citep{ampute}. This complicates any definitive conclusions about imputation method performance in relation to the data generation parameters.

% Missingness mechanisms are usually assumed to be random (MAR; see Table \ref{table:mech} for definitions). However, the missingness mechanisms that are induced in simulation studies do not necessarily match any true missingness generating mechanisms in incomplete empirical data, nor the mechanisms that are typically assumed when imputing such data. On the one hand, there is MCAR missingness, which is unlikely to be the true missingness generating mechanism, but would yield insightful evaluations in a simulation setting. On the other hand, there is MNAR, which is arguably a quite reasonable source of many real-world missingness scenarios but is generally ignored except for those evaluations that are specifically targeted at non-ignorable applications. A disconnect between induced missingness, real missingness, and assumed missingness may result in simulation studies that are not as informative as they could be. 

Missingness should be induced according to several sets of missing data conditions. We encourage simulators to consider different missingness patterns and mechanisms. Missingness mechanisms were first defined in Rubin's seminal work \citep{rubi76}. For more recent considerations see e.g. \citet{seam13, meal15, dore18, more18, scho18, litt20, moha21, scho21}. Although not every missingness mechanism is realistically assumed in practice, they can all offer valuable insights as simulation condition. A disconnect between induced missingness, real missingness, and assumed missingness may result in simulation studies that are not as informative as they could be.

[TODO: add that focusing on a single mechanism (even if the mechanism suits the study aim) is wasteful?] Truly random missingness across all data entries (`missing completely at random' or MCAR under Rubin's definition) may be considered a necessary simulation condition for the evaluation of imputation procedures, since the statistical properties of the observed data given the missing data are known. Any imputation routine that cannot at least mimic the performance of the observed data inference should be deemed inefficient in the scope of the simulation. If an imputation method is not able to solve the problem (i.e. yield valid inference) under MCAR, the statistical properties of the procedure are not universally sound. Sadly, the straightforward case of MCAR is often neglected from simulation studies, while it offers an informative reference condition in many simulation setups.

[TODO: maybe add definitions of missingness mechanisms back in???] A straightforward technique for inducing univariate MAR missingness is described in \citet[][\S 3.2.4]{buur18}, generalizations to multivariate MAR missingness can be found in \citet{ampute}. If the missingness is to be induced in longitudinal data autoregressive MAR models can be useful \citep[see e.g.][model 2 and model 3]{shara2015randomly}. It is advisable to investigate varying shapes of MAR missingness to achieve a more realistic indication of the robustness of the imputation performance across the range of random missingness. The effects of different types of MAR mechanisms are described in \citet{scho18}. Inducing a MAR mechanism presents the simulator with a choice, namely the type or functional form of the missingness model. Given the simulated data distributions, one random missingness model may be far more disastrous to the observed information than another model \citep{scho18}. This may influence the performance of some (but not necessarily all) imputation routines. For example, inference from hot-deck techniques such as predictive mean matching \citep{little1988missing, rubin1986statistical} may be more severely impacted by large amounts of one-tailed missingness than inference from parametric techniques. It would be a shame to overlook such results due to the focus on a single functional form of the missingness-generating model.

Although MAR missingness is often considered as a simulation condition, the problem of spurious MAR is generally overlooked. With MAR missingness mechanisms, observed relations in the data are used to induce missingness during simulation (e.g. weight is made incomplete based on observed gender to emulate a situation wherein one gender is less likely to disclose their weight). These relations in the observed data may, however, be weak or non-existent. If MAR is induced based on weaker relations in the data, claims for a method's applicability to situations where the missingness is random become less valid. The most extreme example would be when MAR is induced from data without multivariate relations. The inferential implications of the missingness would then mimic those of MCAR, even though the missingness is random. In fact, this would amount to one of the special cases under which complete case analysis would be more efficient than imputation \citep[see e.g.][\S 2.7]{buur18}: the missingness does not depend on the incomplete variable. This property might be useful in practice, but considering it as a condition to evaluate performance under MAR missingness is pointless. 

A non-ignorable or MNAR mechanism might fall outside many studies' scope, yet could yield informative insights for daily practice. Even if an imputation method is specifically developed with ignorable missingness in mind, chances are that the method will be applied to non-ignorable empirical missingness at some point in time. After all, for every MNAR mechanism there is a MAR mechanism with equal fit and one cannot definitively verify that empirical missingness is random \citep{molenberghs2008every}. Therefore, it can be argued that MNAR is the more likely mechanism for real-life missingness scenarios. It may be wise to include MNAR missingness in the simulation study just in case. Alternatively, a sensitivity analysis may provide an indication of the validity of the obtained inference, given that the assumed missingness mechanism is suspected to be invalid \citep[see e.g.][part 5]{molenberghs2014handbook}.


%%%%%%%%%%%%%%%%%%%%%%%%%%

% 2a. choose missingness mechanism(s): at least consider MCAR, vary MAR types
% 2b. choose missing data pattern(s): introduce non-monotone multivariate missingness
% 2c. choose missingness proportion(s): at least consider 10, 25 and 50\%

 

% First, one should always consider MCAR missingness (see Table \ref{table:mech} for definitions). This mechanism should be used as a reference condition. Every imputation method must be able to yield valid imputations under MCAR, both in terms of distributional characteristics as well as statistical inference.
% 
% Next, missing data should be induced conform a model that is dependent on the observed data (i.e. a MAR mechanism). Arguably, MAR is the most often-assumed mechanism in practice, and should thus be evaluated carefully. 


In addition to varying the missingness mechanisms, each simulated mechanism should be combined with different missingness patterns. Remember that missingness is only ignorable under MAR when the parameter of the data is distinct and a-priori independent from the parameter of the missing data process. Under MAR missingness we assume that we may use the observed data to make inferences about the joint (observed and unobserved) data. The dependency of the procedure on the assumption under which we obtain inference is only influenced by the amount of missingness. If there is no missingness--or if there is no data, for that matter--the inference does not depend on the assumption. Alternatively, the validity of assumptions becomes increasingly important when the missingness increases. Since we control the MAR mechanism, the assumption under which we may solve the missing data problem should hold and it is only fair to assess performance under stringent missingness conditions. We, therefore, propose to evaluate imputation methodology under several missingness proportions to amulate a realistic range in severity of the missingness problem. Depending on how the missingness mechanism interacts with the simulated data, higher missingness proportions may yield biased or invalid completed data inferences. The missingness proportions should therefore be considered carefully.

[TODO: paraphrase ``be clear which mechanisms are of interest, how they are generated, if they are suitably generated" as conclusion to this section.]
% \begin{table}[htb]
% \begin{center}
% \caption{Suggested univariate missingness proportions. [TODO: change this to 'more than one missingness proportion, depending on the context' and/or add references for the given proportions (e.g. about  that 30\% is a good range for MI).]}
% \label{table:prop}
% \begin{tabular}{lll}
% \hline
%       & Reasoning for suggested missingness proportion \\
% \hline  
% 10\%  & Depending on the size of the data, this percentage can be considered as a lower \\
%       & bound of realistic evaluation. Anything less than 10\% may be of little influence on \\
%       & the true data inference. Performance of a missing data method should at least be \\
%       & acceptable for most missing data problems. \\ \noalign{\smallskip} % & \\
% 25\%  & This is a fair amount of missingness and will, depending on the observed data \\
%       & information, have a noticeable influence on the completed data inference. When \\
%       & compared to the condition with 10\% missingness, the inference obtained under \\
%       & 25\% missingness should be less certain (i.e. confidence/credibility interval width \\
%       & should increase), but estimates should still be properly covered and the statistical \\
%       & properties of the missing data method should be sound. In practice, at least to our \\
%       & at least to our experience in social sciences and official statistics, 25\% \\
%       & univariate missingness can easily be considered as a realistic missingness percentage. \\ \noalign{\smallskip} % & \\
% 50\%  & Performance under 50\% percent simulated missingness will most likely be impacted \\ 
%       & severely. Depending on how the missingness mechanism interacts with the simulated \\
%       & data, some imputation techniques may yield estimates that are under-covered such that \\
%       & the completed data inference should not be deemed valid anymore. If a method yields \\
%       & acceptable inference under 50\% MAR missingness, we can determine that the statistical \\
%       & properties of the imputation methodology are sound. \\
% \hline
% \end{tabular}
% \end{center}
% \end{table}


%%%%%%%%%%%%%%%%%%%%%%%%%%

% \subsection{Imputation methods}

% 3a. Define missing data methods: at least consider complete case analysis as benchmark
% 3b. Choose imputation model parameters: appropriate n_imp and n_it (if applicable), maybe vary predictor matrices?
% 3c. Impute the data
% 3d. Obtain estimates



%%%%%%%%%%%%%%%%%%%%%%%%%%

\subsection{Performance evaluation}

[TODO: make this section less stringent and focus more on the suitability of the performancxe evaluation in the context of the study aim.]

The evaluation criteria used to assess imputation performance vary from one simulator to another. This is not surprising as people from different fields could have a different focus on the problem at hand. There are, however, some overarching issues with assessing imputation method performance. In the first place, there are pitfalls in the evaluation of the estimates that are obtained by fitting the analysis model after imputation (e.g. focusing on one performance measure over another, may yield different conclusions about imputation efficacy). In the second place, diagnostic evaluation of the generated imputations is often left out of simulation study results. Identifying problems with the imputation-generating process (e.g., an iterative imputation algorithm) may offer explanations for under-performance of imputation methods. Such evaluations, however, are typically omitted and valuable insights into the imputation method(s) may then be overlooked. %(i.e. the complete-data model used to estimate the comparative truth). % The choice of performance measures might inadvertently influence simulation results.

At minimum, imputation method performance should be quantified using appropriate measures. It depends on the specifics of each simulation study which performance measures would be most suitable. We recommend to evaluate at least the following points:

(i) The estimates should preferably be unbiased. Note that the way bias is calculated should be carefully chosen and described, since this can greatly influence the interpretation of the results. In most cases, unbiased estimates may be expected under a MCAR missingness mechanism.

(ii) The estimates should have a proper coverage rate. Coverage of a 95\% interval should in theory be $\geq 95$, where a coverage rate of 95\% would be most efficient. Under-coverage (when estimation is too liberal and intervals are too narrow) indicates that the procedure is not confidence valid and may lead to invalid inference. Over-coverage (when estimation is too conservative and intervals are too wide) tells us that efficiency could still be gained. 

% "Note that Neyman's original description of confidence intervals defined the property of randomisation validity as exactly 100(1_alpha) of intervals containing theta.44-46 Confidence validity is the property that the true percentage is at least 100(1_alpha). This latter definition is less well known than the former, with the result that overcoverage and undercoverage are sometimes regarded as similarly bad.47 Of course, randomisation validity would usually be preferred over confidence validity because it implies shorter intervals, but this is not always the case. There are examples of procedures that return both shorter intervals and higher coverage.45, 46" ADEMP paper

(iii) The width of the confidence or credible interval may convey statistical efficiency, which should be considered to compare imputation methods. Wider intervals are associated with more uncertainty whereas a more narrow interval that is still properly covered indicates a sharper inference. However, inference from a wider interval that is properly covered is to be considered more valid than a more narrow interval that is not properly covered anymore. 

(iv) Predictive accuracy may be quantified using the root mean squared error (RMSE). We do not generally recommend the use of the RMSE as evaluation criterion, because this metric does not account for the inherent uncertainty of the missing values \citep[][\S 2.5]{buur18}. However, if a study is aimed at obtaining predictions or classification and inferential validity is not of interest, the RMSE of the predictions may yield valuable information about the methods' efficiency in terms of both accuracy and precision. [TODO: make this an optional bullet point for prediction only: out of sample mean squared prediction error can be compared between methods.]

[TODO: intrgrate this paragraph into the above!] Performance measures should express how well the obtained estimates approximate the comparative truth, but not every metric is suited for the evaluation of imputation methodology. For example, using measures of predictive accuracy or the RMSE (root mean squared error) as the evaluation criterion may increase the rate of false positives \citep[][\S 2.6]{buur18}. And, under certain simulation conditions, an otherwise useful metric such as relative bias can distort the interpretation of results (i.e., a negligible absolute bias may yield an infinite relative bias if the true value of the estimand is zero). The usefulness of evaluation criteria may thus depend on the simulation goal and estimand.

If the goal is inference--not prediction--then the uncertainty about estimates needs to be properly quantified. To capture this uncertainty, the standard errors of the estimates should be correctly calculated, which requires multiple imputation. The aim of multiple imputation is not to reproduce the data, but to allow for obtaining valid inference given that the data are incomplete. This means that, given the framework provided by \citet{rubi87}, statistical properties such as bias, confidence intervals, and the coverage rate of the confidence intervals should be studied. After all, the 95\% confidence interval should contain the `true' value at least 95 out of 100 times \citep[][p. 591]{neym34}. [TODO: add nuance about coverage rate > 95\% irt CIW and efficiency.]

The choice of performance measures may inadvertently distort statistical properties of the imputed data. Developers often only inquire about the `accuracy' (i.e. how well can the method reproduce the original data). Merely focusing on the discrepancy between the true and imputed data would open the door for invalid inferences. 

Next to imputation method performance, the imputation-generating process may be evaluated. Omitting further evaluation of the imputation-generating process may yield sub-optimal imputations. The evaluation of simulated results relies on the relation between missing data models, imputation models, and analysis models. When an imputation model is able to capture the essence of the true non-response mechanism relative to the analysis model, the models are said to be congenial \citep{meng94}. [TODO: change this to being able to embed the analysis procedure and the imputation model within a Bayesian model; the nonresponse mdel may be misspecified while the imputation procedure is congenial.] The congeniality of all imputation models should be assessed, but methods to assess the suitability of imputation models and diagnose misfit often rely on visual inspection of the imputations \citep[see e.g.][]{abayomi2008diagnostics, bond16}. This makes its assessment an infeasible endeavor in many simulation studies. Moreover, the performance of imputation procedures on distributional properties is often ignored too. Even though the estimates on the analysis level may be justified, some methods can yield imputations that may seem completely invalid to applied researchers. For example, one could very accurately estimate average human height by filling in negative values and values that are unrealistically large. While the obtained inference could still be valid under such imputations, the plausibility of the imputed values given the observed data should be under scrutiny. Under many circumstances, imputation methods may be realistically expected to preserve both marginal and conditional distributions with respect to the comparative truth. Imputation methods that fail to do so, should not be considered general-purpose methods (e.g., mean imputation).

[TODO: add segway to this paragraph.] Many contemporary imputation techniques rely on iterative algorithms, such as the Gibbs sampler, to generate imputations. As with any iterative algorithm--but especially with imputation algorithms that are critically considered to be possibly incompatible Gibbs samplers \citep[PIGS,][]{li2012imputing}--algorithmic convergence should be carefully evaluated. Unfortunately, there is no universal quantitative method to diagnose non-convergence in iterative imputation algorithms \citep{zhu15, ober21} and the alternative \citep[visual inspection of the imputation algorithm;][]{buur18} is neither efficient nor failproof. As a result, imputation algorithms may be terminated before reaching a stable state, which could yield sub-optimal imputations and under-estimated performance of the method. Problems with producing stable imputations is not an exclusive quality iterative imputation algorithms. Any method may run into failures of the imputation-generating process and subsequently lack results (e.g. due to over-parameterization errors).

[TODO: integrate this into text above!] Every imputation workflow should contain an evaluation of the obtained imputations. Even though inspecting each imputation may be labor-intensive due to the number of imputations generated in a simulation study, we highly recommend to consider the following aspects:

(i) The absence of non-convergence in the imputation-generating process is a minimum requirement for any imputation method. If non-convergence is suspected, the inference resulting from the imputations might be invalid. However, preliminary work suggests that iterative imputation algorithms could achieve inferential validity before reaching a stable state \citep{ober21}. 

(ii) The fit of the imputation model may be verified with the help of a posterior predictive check \citep[][]{nguy17, zhao22}. A straightforward posterior predictive check for imputation methodology is the multiple over-imputation of observed data values. If the statistical properties of the over-imputed values are equivalent to those of the observed data values, one could infer that the imputation model fits the observed part of the incomplete data reasonably well. Then, by extension, one could assume that the imputation model might be able to produce good imputations for the missing part of the data too. [TODO: add that this is implemented in amelia.]

(iii) The distributional characteristics of the imputations should be inspected for anomalies. The distribution of the incomplete data may differ greatly from the observed data. Under anything but the MCAR assumption, this can be expected. When evaluating imputations, the distributional shapes should be checked and diagnostic evaluations should be performed \citep[see][for a detailed overview of diagnostic evaluation for multivariate imputations]{abayomi2008diagnostics}. When anomalies are found, and if the imputation method is valid, there should be an explanation, especially in the controlled environment of a properly executed simulation study. 

(iv) Finally, the plausibility of the imputed values may be evaluated. Plausible imputations--imputations that could be real values if they had been observed--are not a necessary condition for obtaining valid inference. However, in practice, especially when the imputer and the analyst are different persons, plausible imputations may be a desired property. One would prefer an imputation technique to yield both valid inference and plausible imputations. It should be studied if an imputation method is prone to deliver such impractical results, and if so, under what conditions. When evaluating imputation routines, the evaluator should mention whether the routine is prone to deliver implausible values. [TODO: add nuance here because valid inference, not plausible imputed values, should be the aim (although plausibility is nice).]

[TODO: add segway to reference method paragraph and add that performance on the complete data ofefrs an upper limit for performance.] In addition, the simulator should always perform complete case analysis. We know the theoretical properties of complete case analysis, which makes the technique useful as a point of departure when evaluating imputation performance. Complete case analysis may, therefore, serve as a benchmark method against which imputation performance should be evaluated. Moreover, pairing complete case analysis with MCAR and MAR mechanisms allow the simulator to evaluate the validity of the missing data generation process under the chosen analysis model.


%%%%%%%%%%%%%%%%%%%%%%%%%%


% 4a. Evaluate imputations: check failed methods, algorithmic convergence (if applicable)
% 4b. Check imputation model fit: anomalies in distributional characteristics, plausibility of imputed values (if requested), PPC




%%%%%%%%%%%%%%%%%%%%%%%%%%


% 4c. Apply performance measures: at least consider bias, coverage, CI length, if applicable also RMSE (at prediction and/or cell level)
% fraction of missing information (= model parameter), variance inflation


Last, when the simulator is on the verge of drawing conclusions about the performance of the imputation methodology, the performance should be carefully qualified. Comparing the performance of an imputation routine given a population (or true) parameter allows for quantitative evaluation. Yet, in order to pose qualitative statements about the performance on simulated conditions, comparative methodology is required. For example, when claiming that imputation performance is unacceptable when deviations from normality become rather stringent, such performance is highly dependent on the simulation conditions that are used. For a well-balanced judgment about the severity of the performance drop, comparative simulations with e.g. nonparametric models should be executed. A method may perform badly, but if it still outperforms every other approach, it may yet be of great practical relevance.

% It could be convincing to demonstrate a method's applicability to real-world missing data problems. This can, for example, be achieved by obtaining and imputing a fully observed set of variables, wherein the missingness is mimicked from a similar, incomplete set. Alternatively, an incomplete set could be obtained and truth could be established by removing the incomplete cases from the data. However, the real-world missingness would then be omitted. 


%%%%%%%%%%%%%%%%%%%%%%%%%%

% \subsection{Reporting}


% 
% \begin{table}[tb]
% \begin{center}
% \caption{Steps to consider in imputation simulation studies.}
% \label{table:steps}
% \begin{tabular}{lll}
% \hline
%                         & Simulation aspects under consideration \\
% \hline  
% 1. Set scope            & Aim(s), missing data method(s) to evaluate, simulation design \\
% 2. Obtain truth         & Data-generating mechanism(s), estimand(s), sampling variance \\
% 3. Induce missingness   & Missingness mechanism(s), missing data pattern(s) \\
% 4. Apply methods        & Imputation method(s), analytic method(s) \\
% 5. Evaluate imputations & Imputation-generating process, imputation model fit \\
% 6. Evaluate performance & Performance measures, qualitative judgment \\
% 7. Report               & Text, visualization, checklist, simulation script \\
% \hline
% \end{tabular}
% \end{center}
% \end{table}




%%%%%%%%%%%%%%%%%%%%%%%%%%
%% DISCUSSION
%%%%%%%%%%%%%%%%%%%%%%%%%%

\section{Suggested course of action}

After the careful consideration and execution of all simulation aspects, the evaluations should be properly reported. The inconsistent display of simulation conditions may impact the objectivity of meta-evaluations over imputation methods, as one method's performance may appear to be favorable because of less stringent simulation conditions. This ultimately may lead to statisticians recommending a less efficient method to applied researchers, thereby limiting the efficiency of the imputation approach and unnecessarily lowering the statistical power.

We encourage simulators to document their choices and to be explicit in their descriptions. For example, the simulation design may be presented textually, in a flow chart or as a block of pseudo-code, whereas missingness mechanisms may be written as a function of the data or displayed graphically. Ideally, this should be supplemented by an online repository with all of the data and code required to reproduce the simulation results. To aid simulators in reporting and move towards standardization in evaluation, we provide a draft version for reporting guidelines in Appendix A.1 (also available from \underline{www.gerkovink.com/evaluation}). We invite the readers of this paper to contribute to its development. 

We aim to elicit critical thinking about incomplete data simulation and to establish a common ground for the evaluation of imputation routines. Such a common ground would be the basis of a standardized evaluation. This would allow for fairer and more efficient comparisons between imputation techniques. Ultimately, it would be desirable to evaluate every imputation routine against the same standardized set in order to quantify the statistical properties across imputation routines. If properly executed, such evaluations would allow for careful matching of imputation methodologies to new missing data problems. 


%%%%%%%%%%%%%%%%%%%%%%%%%%
%% APPENDICES
%%%%%%%%%%%%%%%%%%%%%%%%%%

\begin{acknowledgement}
We thank the Amices team for the fruitful discussions.
\end{acknowledgement}
\vspace*{1pc}

\noindent {\bf{Conflict of Interest}}

\noindent {\it{The authors have declared no conflict of interest.}}

\section*{Appendix}

\subsection*{A.1.\enspace Reporting guidelines}

Table \ref{table:check} provides a checklist for reporting on imputation methodology evaluations. 

\begin{table}[ht]
\caption{Checklist for reporting on imputation methodology evaluations.}
\label{table:check}
\begin{tabular}{ll}
% \hline
%   & Suggested aspects to report                                                       \\
\hline
1 & Simulation scope                                                                   \\ \cline{2-2}
$\square$  & Aim                                                                                \\
$\square$  & Design (incl. pseudo-code or flow diagram)                                         \\
$\square$  & Number of simulation repetitions                                                   \\ \hline
2 & Comparative truth                                                                  \\ \cline{2-2}
$\square$  & Data source (model-based or design-based)                                           \\
$\square$  & Sampling variance                                                                  \\
%  & Data characteristics (number of observations, number of variables, variable types, \\
%  & coherence between variables incl. data structures e.g. clustering)                 \\
$\square$  & Data characteristics (incl. multivariate relations and structures e.g. clustering) \\
$\square$  & Estimand                                                                           \\ \hline 
3 & Induced missingness                                                                \\ \cline{2-2}
$\square$  & Missingness mechanism (incl. type or functional form of the missing data   model)  \\
$\square$  & Missingness pattern (incl. missingness proportion)                                 \\ \hline 
4 & Applied methods                                                                    \\ \cline{2-2}
$\square$  & Imputation methods (incl. parameters e.g. the number of imputations)               \\
$\square$  & Analytic methods (incl. calculation of standard errors e.g. pooling   rules)       \\
$\square$  & Reference method (e.g. complete case analysis)                                    \\ \hline 
5 & Imputation evaluation                                                              \\ \cline{2-2}
$\square$  & Imputation-generating process (e.g. algorithmic non-convergence)                   \\
$\square$  & Imputation model fit (e.g. posterior predictive checks)                            \\
$\square$  & Distributional characteristics (e.g. plausibility of imputed values)               \\ \hline 
6 & Performance evaluation                                                             \\ \cline{2-2}
$\square$  & Statistical properties (e.g. confidence validity)                                  \\
$\square$  & Comparative performance (e.g. predictive accuracy)                                 \\ \hline
\end{tabular}
\end{table}


%%%%%%%%%%%%%%%%%%%%%%%%%%
%% REFERENCES
%%%%%%%%%%%%%%%%%%%%%%%%%%
\bibliography{bibliography}
\bibliographystyle{apalike}

% \begin{thebibliography}{10}
% \bibitem[Bauer and Bauer(1994)Bauer, P. and Bauer, M.M.]{bib1}Bauer, P. and Bauer, M. M. (1994). Testing equivalence simultaneously for location and dispersion of two normally distributed populations. \textit{Biometrical Journal} \textbf{36}, 643--660.
% \bibitem[Farrington, C. P. and Andrews, N. (2003)]{bib2}Farrington, C.P. and Andrews, N. (2003). Outbreak detection:
% Application to infectious disease surveillance. In: Monitoring the Health of Populations (eds. R. Brookmeyer and D. F. Stroup), Oxford University Press, Oxford,\break 203--231.
% \bibitem[Rencher(1998)Rencher, A.C.]{bib3}Rencher, A. C. (1998). \textit{Multivariate Statistical Inference and Applications}. Wiley, New York. 
% \end{thebibliography}
% \newpage
% \phantom{aaaa}
\end{document}