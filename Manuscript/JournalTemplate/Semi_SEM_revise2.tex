\documentclass[bimj,fleqn]{w-art}
\usepackage{times}
\usepackage{w-thm}
\usepackage[authoryear]{natbib}
\setlength{\bibsep}{2pt}
\setlength{\bibhang}{2em}
\newcommand{\J}{J\"{o}reskog}
\newcommand{\So}{S\"{o}rbom}
\newcommand{\bcx}{{\bf X}}
\newcommand{\bcy}{{\bf Y}}
\newcommand{\bcz}{{\bf Z}}
\newcommand{\bcu}{{\bf U}}
\newcommand{\bcv}{{\bf V}}
\newcommand{\bcw}{{\bf W}}
\newcommand{\bci}{{\bf I}}
\newcommand{\bch}{{\bf H}}
\newcommand{\bcb}{{\bf B}}
\newcommand{\bcr}{{\bf R}}
\newcommand{\bcm}{{\bf M}}
\newcommand{\bcf}{{\bf F}}
\newcommand{\bcg}{{\bf G}}
\newcommand{\bcs}{{\bf S}}
\newcommand{\bca}{{\bf A}}
\newcommand{\bcd}{{\bf D}}
\newcommand{\bcc}{{\bf C}}
\newcommand{\bce}{{\bf E}}
\newcommand{\ba}{{\bf a}}
\newcommand{\bb}{{\bf b}}
\newcommand{\bc}{{\bf c}}
\newcommand{\bd}{{\bf d}}
\newcommand{\bx}{{\bf x}}
\newcommand{\by}{{\bf y}}
\newcommand{\bz}{{\bf z}}
\newcommand{\bu}{{\bf u}}
\newcommand{\bv}{{\bf v}}
\newcommand{\bh}{{\bf h}}
\newcommand{\bl}{{\bf l}}
\newcommand{\be}{{\bf e}}
\newcommand{\br}{{\bf r}}
\newcommand{\bw}{{\bf w}}
\newcommand{\de}{\stackrel{D}{=}}
\newcommand{\bt}{\bigtriangleup}
\newcommand{\bfequiv}{\mbox{\boldmath $\equiv$}}
\newcommand{\bmu}{\mbox{\boldmath $\mu$}}
\newcommand{\bnu}{\mbox{\boldmath $\nu$}}
\newcommand{\bxi}{\mbox{\boldmath $\xi$}}
\newcommand{\btau}{\mbox{\boldmath $\tau$}}
\newcommand{\bgamma}{\mbox{\boldmath $\Gamma$}}
\newcommand{\bphi}{\mbox{\boldmath $\Phi$}}
\newcommand{\bfphi}{\mbox{\boldmath $\varphi$}}
\newcommand{\bfeta}{\mbox{\boldmath $\eta$}}
\newcommand{\bpi}{\mbox{\boldmath $\Pi$}}
\newcommand{\bequiv}{\mbox{\boldmath $\equiv$}}
\newcommand{\bvarepsilon}{\mbox{\boldmath $\varepsilon$}}
\newcommand{\btriangle}{\mbox{\boldmath $\triangle$}}
\newcommand{\bdelta}{\mbox{\boldmath $\Delta$}}
\newcommand{\beps}{\mbox{\boldmath $\epsilon$}}
\newcommand{\btheta}{\mbox{\boldmath $\theta$}}
\newcommand{\balpha}{\mbox{\boldmath $\alpha$}}
\newcommand{\bsphi}{\mbox{\boldmath $\varphi$}}
\newcommand{\bsig}{\mbox{\boldmath $\sigma$}}
\newcommand{\bfpsi}{\mbox{\boldmath $\psi$}}
\newcommand{\bfdelta}{\mbox{\boldmath $\delta$}}
\newcommand{\bsigma}{{\bf \Sigma}}
\newcommand{\bzero}{{\bf 0}}
\newcommand{\bpsi}{\mbox{\boldmath $\Psi$}}
\newcommand{\bep}{\mbox{\boldmath $\epsilon$}}
\newcommand{\bomega}{\mbox{\boldmath $\Omega$}}
\newcommand{\bfomega}{\mbox{\boldmath $\omega$}}
\newcommand{\blambda}{\mbox{\boldmath $\Lambda$}}
\newcommand{\bflambda}{\mbox{\boldmath $\lambda$}}
\newcommand{\bfsigma}{\mbox{\boldmath $\sigma$}}
\newcommand{\bfpi}{{\mbox{\boldmath $\pi$}}}
\newcommand{\bupsilon}{\mbox{\boldmath $\upsilon$}}
\newcommand{\obs}{{\rm obs}}
\newcommand{\mis}{{\rm mis}}
\theoremstyle{plain}
\newtheorem{criterion}{Criterion}
\theoremstyle{definition}
\newtheorem{condition}[theorem]{Condition}
\usepackage[]{graphicx}
\chardef\bslash=`\\ % p. 424, TeXbook
\newcommand{\ntt}{\normalfont\ttfamily}
\newcommand{\cn}[1]{{\protect\ntt\bslash#1}}
\newcommand{\pkg}[1]{{\protect\ntt#1}}
\let\fn\pkg
\let\env\pkg
\let\opt\pkg
\hfuzz1pc % Don't bother to report overfull boxes if overage is < 1pc
\newcommand{\envert}[1]{\left\lvert#1\right\rvert}
\let\abs=\envert

\begin{document}
%\DOIsuffix{bimj.DOIsuffix}
\DOIsuffix{bimj.200100000}
\Volume{52}
\Issue{61}
\Year{2010}
\pagespan{1}{}
\keywords{Key word one; Key word two; Key word three; Key word four; Key word five;\\
\noindent \hspace*{-4pc} {\small\it (Up to five keywords are allowed and should be given in alphabetical order. Please capitalize the key}\\
\hspace*{-4pc} {\small\it words)}\\[1pc]
\noindent\hspace*{-4.2pc} Supporting Information for this article is available from the author or on the WWW under\break \hspace*{-4pc} \underline{http://dx.doi.org/10.1022/bimj.XXXXXXX} (please delete if not
applicable)
}  %%% semicolon and fullpoint added here for keyword style

\title[Running title]{Long title {\small\it (please use sentence case, e.g. Score tests for exploring complex models)}}
%% Information for the first author.
\author[First Author {\it{et al.}}]{First Author\footnote{Corresponding author: {\sf{e-mail: author@emailaddress.com}}, Phone: +00-999-999-999, Fax: +00-999-999-999}\inst{,1}} 
\address[\inst{1}]{First address {\it(please include the department and the postal address)}}
%%%%    Information for the second author
\author[dd]{Second Author\inst{1,2}}
\address[\inst{2}]{Second address {\it{(please include the department and the postal address)}}}
%%%%    Information for the third author
\author[]{Third Author\inst{2} {\small\it (please provide full author names. Middle names should be indicated by initials only, i.e. Henry J. James)}}
%%%%    \dedicatory{This is a dedicatory.}
\Receiveddate{zzz} \Reviseddate{zzz} \Accepteddate{zzz} 

\begin{abstract}
Start. This is the abstract. You may use this LaTeX template to write your text.
The abstract should not contain multiple paragraphs, formulae NOR references and should ideally
not be longer than 250 words. This is the abstract. You may use this LaTeX template to write your text.
The abstract should not contain multiple paragraphs, formulae NOR references and should ideally
not be longer than 250 words.  This is the abstract. You may use this LaTeX template to write your text.
The abstract should not contain multiple paragraphs, formulae NOR references and should ideally
not be longer than 250 words.  This is the abstract. You may use this LaTeX template to write your text.
The abstract should not contain multiple paragraphs, formulae NOR references and should ideally
not be longer than 250 words.  This is the abstract. You may use this LaTeX template to write your text.
The abstract should not contain multiple paragraphs, formulae NOR references and should ideally
not be longer than 250 words. Start. This is the abstract. You may use this LaTeX template to write your text.
The abstract should not contain multiple paragraphs, formulae NOR references and should ideally
not be longer than 250 words. 
 Stop.
\end{abstract}



%% maketitle must follow the abstract.
\maketitle                   % Produces the title.

%% If there is not enough space inside the running head
%% for all authors including the title you may provide
%% the leftmark in one of the following three forms:

%% \renewcommand{\leftmark}
%% {First Author: A Short Title}

%% \renewcommand{\leftmark}
%% {First Author and Second Author: A Short Title}

%% \renewcommand{\leftmark}
%% {First Author et al.: A Short Title}

%% \tableofcontents  % Produces the table of contents.


\section{Introduction}

This is normal body text. Please note that the body text must be divided into numbered sections with
suitable short verbal titles. Sub-headings are allowed but not sub-subheadings. Please use sentence case in the
headings.

\subsection{Second level heading}

This is the body text. Please note that cross-references in the body text should be shown as follows:
(Miller, 1900), (Miller and Baker, 1900) or if three or more authors (Miller {\it{et al}}., 1900)
\vspace*{12pt}

\noindent Bullet lists are not allowed. Always use (i), (ii), etc.
\vspace*{12pt}

\noindent Sentences should never start with a symbol.
\vspace*{12pt}

\noindent Names of software packages and website addresses should be written in {\tt{Courier new, i.e. Stata, the R package
MASS, http://www.biometrical-journal.com.}}


\begin{table}[htb]
\begin{center}
\caption{The caption of a table.}
\begin{tabular}{lll}
\hline
Description 1 & Description 2 & Description 3\\
\hline
Row 1, Col 1 & Row 1, Col 2 & Row 1, Col 3\\
Row 2, Col 1 & Row 2, Col 2 & Row 2, Col 3\\
\hline
\end{tabular}
\end{center}
\end{table}
\begin{equation}
\left({\theta^{0}_{i}}\atop{\theta^{1}_{i}}\right) \sim N(\theta,\Sigma),\quad {\mathrm{with}}\ 
{{\theta}} = \left({\theta_{0}}\atop{\theta_{1}}\right)\ {\mathrm{and}}\ \Sigma =
\left(\begin{array}{cc}
\sigma^{2}_{0} & \rho\sigma_{0}\sigma_{1}\\
\rho\sigma_{0}\sigma_{2} & \sigma^{2}_{1}
\end{array}\right).
\end{equation}

\noindent This is the body text. Only number equations which are referred to in the text body. If equations
are numbered, these should be numbered continuously throughout the text. Not section wise! Please
carefully follow the rules for mathematical expressions in the ``Instructions to Authors''.

\begin{figure}[htb]
\begin{center}
\includegraphics[bb= 0 0 115 87]{empty.eps}
\caption{The figure caption ($b_{2}b_{2}a^{n}a^{n}$).}
\end{center}
\end{figure}
\begin{acknowledgement}
An acknowledgement may be placed at the end of the article.
\end{acknowledgement}
\vspace*{1pc}

\noindent {\bf{Conflict of Interest}}

\noindent {\it{The authors have declared no conflict of interest. (or please state any conflicts of interest)}}

\section*{Appendix {\it(please insert here, if applicable)}}

\subsection*{A.1.\enspace Second level heading}

Please insert appendices before the references.

\begin{thebibliography}{10}
\bibitem[Bauer and Bauer(1994)Bauer, P. and Bauer, M.M.]{bib1}Bauer, P. and  Bauer, M. M. (1994). Testing equivalence simultaneously for location and  dispersion of two normally distributed populations.  \textit{Biometrical  Journal} \textbf{36}, 643--660.
\bibitem[Farrington, C. P. and Andrews, N. (2003)]{bib2}Farrington, C.P. and Andrews, N. (2003). Outbreak detection:
Application to infectious disease surveillance. In: Monitoring the Health of Populations (eds. R. Brookmeyer and D. F. Stroup), Oxford University Press, Oxford,\break 203--231.
\bibitem[Rencher(1998)Rencher, A.C.]{bib3}Rencher, A. C. (1998).  \textit{Multivariate Statistical Inference and Applications}. Wiley, New  York. 
\end{thebibliography}
\newpage
\phantom{aaaa}
\end{document}